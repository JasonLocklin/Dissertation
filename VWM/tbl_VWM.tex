% latex table generated in R 3.1.1 by xtable 1.7-4 package
% Sun Mar 22 13:42:01 2015
\begin{sidewaystable}[p]
\centering
\begin{tabular}{rrllrrrrlr}
  \hline
 & Age & Sex & Handedness & CES & VWM(1) & VWM(2/3) & Stars & Copying & Bisection \\ 
  \hline
487 &  61 & F & Right & 23.0 & 0.15 & 0.04 & 0& + & 2.2 \\ 
  35 &  51 & F & Right & 27.0 & 0.15 & 0.04 & 17& + & 0.1 \\ 
  489 &  66 & M & Left & 31.0 & 0.25 & 0.08 & 0& + & 1.0 \\ 
  171 &  71 & F & Left & 112.0 & 0.13 & 0.00 & 0& -  & 1.4 \\ 
  454 &  70 & M & Right & 221.5 & 0.23 & 0.17 & 0& + & 6.3 \\ 
  213 &  65 & F & Right & NA & 0.2 & 0.30 & 100& + & 7.3 \\ 
  396 &  85 & M & Right & NA & 0.3 & 0.55 & 87& + & 8.1 \\ 
  465 &  63 & F & Right & NA & 0.3 & 0.45 & 97& + & 12.9 \\ 
   \hline
\end{tabular}

\caption{Table presents demographic data, measures of attention (CES)
	and visual working memory, as well as performance on the three clinical
	measures of neglect by the patient group (described in Results). ``CES''
	indicates the leftward cue-effect-size on the COVAT test, with larger
	numbers indicating more difficulty re-orienting leftward when attention
	is initially cued to non-neglected, right space. ``VWM(1)'' is the
	average probability a patient guesses the target colour in the single
	target condition, with increased values indicating a deficit.
	``VWM(2/3)'' is the average probability a patient selects one of the
	distractor colours, with increasing values indicating a colour-location
	binding deficit. Values for ``Stars'' are coded as the percentage of
	leftward stimuli missed on the Star Cancellation task. Neglect observed
	in figure copying is coded as a ``+'' under ``Copying.'' Line bisection
	performance is recorded as the bias, in terms of percentage of line
	length, with positive values indicating rightward bias, under
``Bisection.''}
\label{tbl_VWM}
\end{sidewaystable}
