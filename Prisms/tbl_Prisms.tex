% latex table generated in R 3.1.1 by xtable 1.7-4 package
% Sun Mar 22 20:15:21 2015
\begin{sidewaystable}[p]
\centering
\begin{tabular}{rrllllllll}
  \hline
 & Age & Sex & Handedness & Star(pre) & Star(post) & Bell(pre) & Bell(post) & Copy(pre) & Copy(post) \\ 
  \hline
10 &  68 & M & Right & 93 & 87 & 100 & 89 & + & + \\ 
  27 &  43 & M & Right & 0 & 7 & 6 & 0 & - & - \\ 
  95 &  70 & M & Right & 7 & 0 & 33 & 39 & + & + \\ 
  163 &  68 & F & Left & 30 & 7 & 6 & 29 & + & + \\ 
  97 &  66 & M & Right & 0 & 0 & 0 & 0 & - & - \\ 
  171 &  71 & F & Left & 0 & 0 & 6 & 6 & + & - \\ 
   \hline
\end{tabular}
\caption{Table (a), above, includes demographic information for the
	patients, as well as performance on star cancellation, bell
	cancellation, and figure copying, all before and after prism adaptation
	(See Results for analysis). For star and bell cancellation, values
	indicate the percentage of left-sided targets omitted. For figure
	copying, a ``+'' indicates the presence of neglect. Table (b), below,
	includes performance on the line bisection (LB), temporal estimation
	(TE), and spatial working memory (SWM) tasks. Line bisection is recorded
	as percentage of line length, with positive values indicating rightward
	bias. TE values represent the slope of a linear model of the log-log
	transformed real and estimated time intervals. A value of 1 indicates
	would indicate estimates that increase in proportion to actual time
	intervals. SWM values indicate accuracy based on hits minus false
alarms.}
\label{tbl_Prisms}

\end{sidewaystable}
